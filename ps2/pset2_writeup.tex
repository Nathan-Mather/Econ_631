\documentclass{article}

\usepackage[margin=1in]{geometry} 
\usepackage[fleqn]{mathtools}
\usepackage{amsmath,amsthm,amssymb}
\DeclareMathOperator*{\argmax}{argmax} % thin space, limits underneath in displays
\DeclareMathOperator*{\argmin}{argmin} % thin space, limits underneath in displays
\usepackage{graphicx}
\usepackage[toc,page]{appendix}
\usepackage[square,sort,comma,numbers]{natbib}
\bibliographystyle{acm} %acm, abbrv, ieeetr, plain, unsrt plainnat
\usepackage{listings}
\usepackage{color}
\usepackage{hyperref}
\usepackage{bm}
\usepackage{bbm}

\newcommand{\E}{\mathbb{E}}
\newcommand{\V}{\mathrm{V}}
\newcommand{\N}{\mathcal{N}}
\newcommand{\R}{\mathbb{R}} 
\newcommand{\1}{\mathbbm{1}}


\title{Economics 631 IO - Fall 2019\\Problem Set 2}
\author{Nathan Mather and Tyler Radler}
\date{\today}

\begin{document}
\maketitle

\section{BLP - Random Coefficient}

\subsection*{Preliminaries}
Each firm chooses price to solve the problem

$$\text{max}_{p_j} (p_j - mc_j)Ms_j(\bm p, \bm x, \sigma)$$

The FOC is

$$0 = (p_j - mc_j)M\frac{\partial s_j}{\partial p_j} + M s_j$$

and so the price will be determined by the following condition:

$$p_j = mc_j-s_j(\frac{\partial s_j}{\partial p_j})^{-1}$$.
 The market share for product $j$ is given by 

$$ s_{j}(\bm p, \bm x, \theta) = \int \frac{\exp (\beta_i x_{j} - \alpha p_{j})} {1 + \sum_{j'} \exp(\beta_i x_{j'} - \alpha p_{j'})} dF(\beta_{i})$$

Given our functional form assumptions we can rewrite $\beta_{i} = \beta + \sigma v_{i}$ where $\beta$ is the mean of the distribution and $\sigma$ is the standard deviation, $v_i \sim \mathcal{N}(0,1)$. Additionally we can define the mean utility of purchasing product $j$ as $\delta_j = \beta x_{j} - \alpha p_j$ and rewrite the market share expression in terms of $\delta_j$ and $\sigma$.

$$ s_{j}(\bm p, \bm x, \bm \delta, \sigma) = 
\int \frac{\exp (\delta_j  + \sigma x_j v_{i})}
{1 + \sum_{j'} \exp(\delta_{j'} + \sigma x_j' v_{i})}
dF(v)
$$
	
Note that if we rewrite the above expression as $$s_{j}(\bm p, \bm x, \bm \delta, \sigma) = 
\int \tilde{s}_j (\bm p, \bm x, \bm \delta, \sigma)  dF(v_{i}) $$ we can get a fairly-nice expression for the own-price derivative with respect to the price:

$$\frac{\partial s_j}{p_j} = \int (-\alpha)\frac{\partial \tilde{s}_j}{\partial \delta_j}dF(v) = -\alpha \int \tilde{s}_j(1-\tilde{s}_j)dF(v)$$

where the last equality is due to properties of the logit error. Thus our final price condition is 

$$p_j = mc_j - \int \tilde{s}_j dF(v)[ -\alpha \int \tilde{s}_j(1-\tilde{s}_j)dF(v)]^{-1}$$

\section{Q1}
We are given that $\alpha = 1, \beta = 1, \sigma = 1, x_1 = 1, x_2 = 2, x_3 = 3,$ and $mc_j = x_j$. 

The price vector is: 

\color{red}
write write write
\color{black}

\section{Q2}
Now $\alpha = .5, \beta = .5, \sigma = .5, x_1 = 1, x_2 = 2, x_3 = 3,$ and $mc_j = x_j$. 
The price vector is: 

\color{red}
write write write. It is different because blah.
\color{black}


\section{Q3}
After the merger the profit maximization problem for the new firm is  

$$\text{max}_{p_1, p_2} (p_1 - mc_1)s_1(\bm p, \bm x, \sigma) + (p_2 - mc_2)s_2(\bm p, \bm x, \sigma)$$

The FOC for $p_1$ is 

$$0 = s_1 + (p_1 - mc_1)\frac{\partial s_1}{\partial p_1}   + (p_2 - mc_2)\frac{\partial s_2}{\partial p_1}$$

and so the price will be determined by the following condition:

$$p_1 = mc_1 - (s_1 + (p_2 - mc_2)\frac{\partial s_2}{\partial p_1})(\frac{\partial s_1}{\partial p_1})^{-1}$$

Note that using our previous notation,
$$\frac{\partial s_2}{\partial p_1} = - \alpha \int \tilde{s}_1\tilde{s}_2dF(v)$$

The FOC for $p_2$ is symmetric to that of $p_1$, and thus the optimal $p_1$ and $p_2$ are given by

$$p_1 = mc_1 - [\int \tilde{s}_1 dF(v) + (p_2 - mc_2)(- \alpha \int \tilde{s}_1\tilde{s}_2dF(v))][-\alpha \int \tilde{s}_1(1-\tilde{s}_1)dF(v)]^{-1}$$


$$p_2 = mc_2 - [\int \tilde{s}_2 dF(v) + (p_1 - mc_1)(- \alpha \int \tilde{s}_2\tilde{s}_1dF(v))][-\alpha \int \tilde{s}_2(1-\tilde{s}_2)dF(v)]^{-1}$$

As firm 3 has not merged it's optimal price condition is the same:

$$p_3 = mc_3 - \int \tilde{s}_3 dF(v)[ -\alpha \int \tilde{s}_3(1-\tilde{s}_3)dF(v)]^{-1}$$


\section{Q4}
The change in consumer surplus is given by the compensating variation, which we can calculate as follows:

$$CV_i = \frac{\text{log}(\sum_j \text{exp}(V_{ij}^{old}) - (\sum_j \text{exp}(V_{ij}^{new})}{\alpha}$$

where $V_{ij} = \beta_ix_j - \alpha p_j$. The total consumer surplus is found by 

$$CV = M*\int CV_i dF(v)$$

where $M$ is the number of people in the market. The change in producer surplus is just the change in profits, and is given by 

$$\Delta \pi = \sum_j (p_j^{new} - mc_j^{new})Ms_j^{new} - (p_j^{olds} - mc_j^{old})Ms_j^{old}$$

and the total change in welfare is 

$$\Delta \text{Surplus} = \Delta \pi - CV = M[(\sum_j (p_j^{new} - mc_j^{new})s_j^{new} - (p_j^{olds} - mc_j^{old})s_j^{old}) - \int CV_i dF(v)] $$

The change in surplus when we normalize $M = 1$ is 

\color{red}
insert ish here
\color{black}

\section{Q5}
If we allow the merging firm's marginal costs to decrease from $x_j$ to $\frac{x_j}{2}$ we get the following pricing equilibrium and change in consumer, producer and total surplus:

\color{red}
insert ish here
\color{black}


\end{document}
